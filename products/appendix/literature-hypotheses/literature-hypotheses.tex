\documentclass[12pt, oneside]{article}   	% use "amsart" instead of "article" for AMSLaTeX format

%%%%%%%%%%%%%%%%%%%%%%%%%%%%%%%%%%%%%%%%%%%%%%%%%%%%
% set up packages
%%%%%%%%%%%%%%%%%%%%%%%%%%%%%%%%%%%%%%%%%%%%%%%%%%%%
%\usepackage[landscape]{geometry}
\usepackage{tabularx}

\usepackage{textcomp}                
\usepackage{amsmath}                
\usepackage{graphicx}                
\usepackage{amssymb}                
\usepackage{fancyhdr}                
\usepackage{bm}                

% package for comments
\usepackage{soul}     
\usepackage{setspace}

%%%%%%%%%%%%%%%%%%%%%%%%%%%%%%%%%%%%%%%%%%%%%%%%%%%%
% call packages
%%%%%%%%%%%%%%%%%%%%%%%%%%%%%%%%%%%%%%%%%%%%%%%%%%%%	
%\geometry{letterpaper, marginparwidth=60pt} % sets up geometry              		
\doublespacing % setspace
	
\usepackage[superscript,noadjust]{cite} % puts dash in citations to abbreviate
%\usepackage [autostyle, english = american]{csquotes} % sets US-style quotes
%\MakeOuterQuote{"} % sets quote style

\usepackage{natbib}
%\bibliographystyle{abbrvnat}
\setcitestyle{authoryear,open={(},close={)}}


%%%%%%%%%%%%%%%%%%%%%%%%%%%%%%%%%%%%%%%%%%%%%%%%%%%%

\begin{document} 

\begin{tabularx}{\textwidth}{|l| >{\setlength{\baselineskip}{.5\baselineskip}}X| >{\setlength{\baselineskip}{.5\baselineskip}}X |}
\hline
Authors & Description & Meristems \\ \hline
 \cite{wyatt1982}       &  "On a more general level, we can consider the evolutionary trade-offs that have shaped inflorescence architecture within the context of plant life histories. Schemske (1980) has suggested that the upper limit to the size of the floral display of Brassavola nodosa is set by a life history framework in which early reproduction is at a premium. The highest level of conflict, then, is that between resource allocation to growth and maintenance vs. reproduction."      &        "On a more general level, we can consider the evolutionary trade-offs that have shaped inflorescence architecture within the context of plant life histories. Schemske (1980) has suggested that the upper limit to the size of the floral display of Brassavola nodosa is set by a life history framework in which early reproduction is at a premium. The highest level of conflict, then, is that between resource allocation to growth and maintenance vs. reproduction."     \\ \hline
  \cite{schemske1980}   &  Low probability of survival to later in the season selects against reproductive delays, even though reproductive delays are associated with larger inflorescences and higher reproductive success  &           \\ \hline
  \cite{stebbins1974} p 277        & Advantage of raceme over compound cyme: as with solitary axillary flowers, production of many, few-flowered axillary racemes distributes reproduction more evenly over the plant as a whole than production of a few, many-flowered terminal inflorescences &           \\ \hline
\cite{stebbins1974} p 277        & Advantage of raceme over compound cyme: synthesis of reproductive material extends over a long period of time, so amount synthesized at any one time is less than for a compound determinate inflorescence &           \\ \hline
\cite{stebbins1974} p 278-9       & Advantage of raceme over compound cyme: "Inherent in an inflorescence that is indeterminate either with respect to its main axis or because of repeated sympodial branching is a flexibility that enables individual plants to form few or many flowers depending upon the length of the favorable season. Consequently, one might expect natural selection to favor this kind of inflorescence in a plant adapted to a semiarid climate having great variation in precipitation from one season to the next."  &           \\ \hline
   \cite{stebbins1974} p 279-80       &  Presence of a terminal flower is associated with less vigorous vegetative growth; weak vegetative growth prevents inhibition of terminal flower and production of indeterminate inflorescence  &           \\ \hline
\end{tabularx}

Stebbins cites examples of phenotypic  plasticity: Pharbitis nil with either indeterminate inflorescences or terminal flowers (Marushige 1965, Bhar and Radforth 1969). Also cites examples from Penstemon corymbosus and discusses how a terminal flower is associated with "less vigorous growth"

\begin{tabularx}{\textwidth}{|l| >{\setlength{\baselineskip}{.5\baselineskip}}X|l|}
\hline
Authors & Description & Meristems \\ \hline
  \cite{} Duffy et al. 1999 /Bonser and Aarssen 2003     &In annual plants, branching maximizes plant size and the number of reproductive meristems at the end of the season &           \\ \hline
  \cite{} Bonser and Aarssen 2003   & responses in meristem allocation should occur at an early age; in high resource environments, this maximizes the number of growing stems accumulating size and meristems early in development, while in low resource environments there should be strong apical dominance/suppression of axillary meristems early to maximize LRS &           \\ \hline
   \cite{scheepens2011}       &  "it could be hypothesized that the indeterminate flowering in C.* carniolica versus the determinate flowering in C.* thyrsoides is due to adaptation to climate. The submediterranean climate could favor indeterminate flowering in C.* carniolica, because this would allow a longer flowering period throughout the long summer until environmental conditions would deteriorate, whereas determinate and fast flowering would be more favourable in short growing season in the high Alps where seed production must be secured before temperatures drop."
  &           \\ \hline
  \cite{} Bonser and Aarssen 2003   & plants may have higher allocation to R in low resource environments vs. high resource environments; this is important when R meristems produce single fruits rather than  inflorescences &           \\ \hline
 \end{tabularx}
 
  \begin{tabularx}{\textwidth}{|l| >{\setlength{\baselineskip}{.5\baselineskip}}X|l|}
\hline
Authors & Description & Meristems \\ \hline

  \cite{}  Bonser and Aarssen 2006      & High branching intensity in high resource environments; allocation to R vs. G in response to relative juvenile and adult survival probability  &           \\ \hline
 \cite{}  Bonser and Aarssen 2006      & Within semelparous species, reproductive architecture matters: inflorescences have greater rep output per rep meristem than species with single flower from rep meristem, so resources may permit a few multiflowered inflorescences (s.t. low value of reproductive meristem allocation) or many single flowers (s.t. high value of reproductive meristem allocation)  &           \\ \hline

 \cite{} Salomonson 1994       & Flexibility in duration of meristem activity helps explain plant's ability to respond to resource availability    &           \\ \hline
  \cite{} Law 1979       &  Tiller number constrains reproduction   &           \\ \hline
  \cite{} Smith 1984      &  Biomass does not determine plant form   &           \\ \hline
  \cite{} Watson 1984      &  Biomass does not determine plant form   &           \\ \hline
  \cite{} Geber 1990      &  Biomass does not determine plant form   &           \\ \hline
\cite{} Cohen 1971       & Model predicts sharp transition from growth to reproduction; this is not observed. One explanation is that there are constraints on how F can vary with time or with plant size; eg. meristems have to growth first    &           \\ \hline

  \end{tabularx}

 \begin{tabularx}{\textwidth}{|l| >{\setlength{\baselineskip}{.5\baselineskip}}X|l|}
\hline
Authors & Description & Meristems \\ \hline
 \cite{} Bonser and Aarssen 1996     & High soil resources, light limiting: selection against high allocation to G or R; high I favored/strong apical dominance for vertical growth   &           \\ \hline
 \cite{} Bonser and Aarssen 1996     & High soil resources, high light: selection against allocation to I; selection for G and growth through branching to take advantage of high light and nutrients; for monocarpic species this maximizes the absolute/relative number of meristems committed to R at the end of the season (cf. Geber 1990 and Smith 1984)   &           \\ \hline
 \cite{} Bonser and Aarssen 1996     & Low soil resources, high light: selection against high G as w low resources there should be slow growth rate and a strong negative tradeoff between growth and reproduction; allocation to growth may mean slow growing plants don't have enough time to reproduce bc slow growth and probability of death; expect higher allocation to R in monocarpic species &           \\ \hline
    \cite{} Bonser and Aarssen 2003   & meristem allocation patterns may be weaker late in life? &           \\ \hline
 \end{tabularx}
 
  \begin{tabularx}{\textwidth}{|l| >{\setlength{\baselineskip}{.5\baselineskip}}X|l|}
\hline
Authors & Description & Meristems \\ \hline
 \cite{} Bonser and Aarssen 2003   & high allocation to G and R meristems in large plants/high resource availability; low allocation to G and R in small plants/low resource availability &           \\ \hline
\cite{} Bonser and Aarssen 2003   & in favorable resource environments, increased allocation to new branches/G meristems is positively correlated with R meristems/negatively with I meristems &           \\ \hline
 \cite{}  Bonser and Aarssen 2006      & Within semelparous species, reproductive architecture matters: so  &           \\ \hline
 \end{tabularx}
 
  \begin{tabularx}{\textwidth}{|l| >{\setlength{\baselineskip}{.5\baselineskip}}X|l|}
\hline
Authors & Description & Meristems \\ \hline
\cite{} Cohen 1971/KR1982       & When length of growing period is unpredictable, there should be some overlap in time of making seeds/leaves  &           \\ \hline
\cite{} Dalgleish and Hartnett 2006     & Perennials but: in arid environments meristem limitation constrains response to resource availability, while mesic grasslands maintain a meristem bud bank and are able to increase production in response to resource pulses; costs to maintain them (Lehtila and Larsson 2005) but benefits may include ability to resprout or ability to activate across the season; in arid environs small scale disturbance and patch dynamics are key while in mesic fire and changes in resource availability are - timing of meristem/resource key &           \\ \hline
\cite{} Kim and Donohue 2012     & Variation in mono/polycarpy in plants is often associated with plant architecture; selection against multiple rosettes contributes to evolution of semelparous strategy &           \\ \hline
\cite{} Korner 2015     & Tissue growth/meristem as sink is more constraining than resource availability &           \\ \hline

 \end{tabularx}
 
   \begin{tabularx}{\textwidth}{|l| >{\setlength{\baselineskip}{.5\baselineskip}}X|l|}
\hline
Authors & Description & Meristems \\ \hline
  \cite{stebbins1974} p 276        & Reduction from cyme to solitary flower is associated with habitats in which rapid flowering is advantageous      &           \\ \hline
 \cite{stebbins1974} p 276        & Solitary flowers in axils are associated with low light, understory habitats where distributing flowers across all of plant is advantageous     &           \\ \hline
  \cite{stebbins1974} p 277        & Selection may favor an inflorescence with the same reproductive capacity but having more flowers, each with fewer seeds, if there is a predatory insect laying eggs on a flower (cf. Burtt 1960)    &           \\ \hline
   \cite{wyatt1982}       &  "the indeterminate growth patterns of a racemose inflorescence might be favored in climates showing low predictability within or between seasons in terms of precipitation or other resources affecting pollination or fruit maturation." This statement is in reference to pollinators.    &           \\ \hline
    \cite{Prusinkiewicz2007}       &  For fixed season lengths, the optimal inflorescence is a panicle. For variable season lengths, racemes and cymes are favored because a plant produces some flowers early but can continue to flower if the season continues to extend. For racemes and cymes, only some of the meristems are in the floral state at a given time and this may lead to higher fitness than panicles. Cites this as a form of beg hedging.  &           \\ \hline

 \end{tabularx}


The evolution of plant
architecture (sussex and kerk

Constraints: FSM plant models (Christophe et al. 2008)

Finley - cost of apical dominance
 Lauxmann - transient carbon limitation
 Obeso - cost of reproduction
 
 Hopper 2003, 
 King and Roughgarden 1982a/b - Graded allocation, vegetative multiple switches, Cohen 1966 - optimizing rep in variable envir, Haccou Iwasa 1995, Kussell Leibler 2005


\iffalse



% \cite{Geber1990} describes how meristem limitation generates positive correlations between growth and fecundity within a life stage, while resource limitation generates negative correlations between growth and fecundity within a life stage due to resource competition.


% I think one way of using these models would be to ask about the optimal strategies under different allocation rules. What is the best strategy when there are no meristems? What is the best strategy when the rates are the same but there are meristems? A model of resource allocation decisions will inevitably produce trade-offs that are the result of resource limitation. Will a model of meristem allocation decisions inevitably produce trade-offs that are the result of meristem limitation? 

% Resource limitation can generate trade-offs. What does this mean? This means that in a situation where resources are limited, we might expect to see trade-offs that are the result of resource allocation decisions (we wouldn't see these decisions when resources are not li mited). Meristem limitation an also generate trade-offs. This means that in a situation where meristems are limited, we might expect to see trade-offs that are the result of meristem allocation decisions (we wouldn't see these decisions when meristems are not limited). 

% Is it possible for me to link within-meristem dynamics to the dynamics of the whole plant? For example, I just looked up papers on within-host and between-host patterns in SIR models. Maybe it's possible to say something about the within-meristem dynamics (growth of individual modules?) and between-meristem dynamics?

% Here's \cite{Geber1990}: "Resource limitation can generate trade-offs between life stages only if current and future expenditures draw on the same limited resource pool, or if current investment patterns affect an organism's ability to gamer resources in the future." I think this means that the resource allocation models generate trade-offs by the second method (current investment affects later resource gathering). Plants that grow big do not deplete their vegetative pool later but instead are able to produce more reproductive biomass. 

% Particular selection regimes produce correlations consistent with a model of resource limitation while other selection regimes produce correlations consistent with a model of meristem limitation. 

Another aspect is that I think there are two potential parameters to manipulate in order to cause resources to be the limiting factor vs. meristems. The primary meristem division process has a rate $\beta_1 - \alpha$ or something like $r$. This is then also the same rate for the primary meristem to inflorescence meristem rate. But this is actually two separate things: one is the within primary meristem and the other between meristem pools. So we might add $ \delta (\beta_1-\alpha)$ to the primary meristem division rate. The product of the per meristem growth $\delta$ and the division rate $\beta_1 - \alpha$ is the is the growth rate. For $\delta = 1$, one meristem produces two meristems. For $\delta < 1$, one meristem divides normally but produces less than the equivalent of two meristems because each meristem does not pay its own cost relative to meristem production. In other words, for $\delta <1$, resources are limiting. For $\delta>1$, one meristem produces more than the equivalent of two meristems - now the rate of primary meristem to inflorescence meristem conversion is limiting.

Ok, here's an alternative. What if I write the probability of symmetric division as $p(t) q(t)$ and the probability of asymmetric division as $(1-p(t)) (1-q(t))$? The probability of symmetric division is $p(t)$ and the proportion of photosynthate that's allocated to vegetative growth is $q(t)$. Either allocating all division or growth to vegetative pool means all the allocated resources in the other direction are 'wasted'. It makes no sense to allocate photosynthate to reproduction unless meristem divisions are starting to support reproduction as well.  Strategies with high meristem division and low photosynthate allocation, or low meristem division and high photosynthate allocation, are identical in a two pool model. The appeal of something like this is that it would make it possible to include resource allocation, meristem allocation, or both in the model. The effects would somehow have to be asymmetric? Stuck again.

% \cite{bonser2006} investigates life history evolution, apical dominance, branching intensity, and reproductive effort. They find that semelparity is associated with greater allocation to reproductive meristems relative to nonreproductive meristems. Iteroparity is associated with greater suppression of axillary meristem development. Branching intensity was unassociated with life history evolution. High reproductive effort in semelparous species is consistent with the idea that natural selection should favor investment in reproduction because there is one opportunity for reproduction. This is consistent with other studeis measuring proportion of resource or biomass allocated to reproduction. In semelparous species, increasing allocation to meristem will increase absolute number of meristems available for reproduction. For iteroparous species, this may increase vegetative storage. High branching is also observed in plants with many resources. 

% \cite{duffy1999} asks whether plant biomass or total number of meristems is a more appropriate currency for understanding relative fitness in plant strategies.

\fi


\bibliographystyle{/Users/gregor/Dropbox/bibliography/styleFiles/ecology} 
\bibliography{/Users/gregor/Dropbox/bibliography/optimal-control-lit}

\end{document}