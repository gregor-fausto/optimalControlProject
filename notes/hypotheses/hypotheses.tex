\documentclass[12pt, oneside]{article}   	% use "amsart" instead of "article" for AMSLaTeX format

%%%%%%%%%%%%%%%%%%%%%%%%%%%%%%%%%%%%%%%%%%%%%%%%%%%%
% set up packages, geometry
%%%%%%%%%%%%%%%%%%%%%%%%%%%%%%%%%%%%%%%%%%%%%%%%%%%%
\usepackage{geometry, textcomp, amsmath, graphicx, amssymb,fancyhdr,subcaption,bm}                
	
\geometry{letterpaper, marginparwidth=60pt}                   		
\usepackage[superscript,noadjust]{cite} % puts dash in citations to abbreviate
%\usepackage [autostyle, english = american]{csquotes} % sets US-style quotes
%\MakeOuterQuote{"} % sets quote style

\usepackage{hyperref}
\hypersetup{
    colorlinks=true,
    linkcolor=blue,
    filecolor=magenta,      
    urlcolor=cyan,
}

\usepackage{etoolbox}
\AtBeginEnvironment{quote}{\small}

\usepackage{float,color}

\usepackage{pgf, tikz, eqnarray}
\usetikzlibrary{arrows, automata}
%%%%%%%%%%%%%%%%%%%%%%%%%%%%%%%%%%%%%%%%%%%%%%%%%%%%

%%%%%%%%%%%%%%%%%%%%%%%%%%%%%%%%%%%%%%%%%%%%%%%%%%%%
\pagestyle{plain}                                                      %%
%%%%%%%%%% EXAFT 1in MARGINS %%%%%%%                                   %%
\setlength{\textwidth}{6.5in}     %%                                   %%
\setlength{\oddsidemargin}{0in}   %% (It is recommended that you       %%
\setlength{\evensidemargin}{0in}  %%  not change these parameters,     %%
\setlength{\textheight}{8.5in}    %%  at the risk of having your       %%
\setlength{\topmargin}{0in}       %%  proposal dismissed on the basis  %%
\setlength{\headheight}{0in}      %%  of incorrect formatting!!!)      %%
\setlength{\headsep}{0in}         %%                                   %%
\setlength{\footskip}{.5in}       %%                                   %%
%%%%%%%%%%%%%%%%%%%%%%%%%%%%%%%%%%%%                                   %%		

%%%%%%%%%%%%%
% DEFINE CODE BLOCK
%%%%%%%%%%%%%
\usepackage{listings}

\definecolor{dkgreen}{rgb}{0,0.6,0}
\definecolor{gray}{rgb}{0.5,0.5,0.5}
\definecolor{mauve}{rgb}{0.58,0,0.82}

\lstset{frame=tb,
  language=R,
  aboveskip=3mm,
  belowskip=3mm,
  showstringspaces=false,
  columns=flexible,
  basicstyle={\small\ttfamily},
  numbers=none,
  numberstyle=\tiny\color{gray},
 % keywordstyle=\color{blue},
  commentstyle=\color{dkgreen},
  stringstyle=\color{mauve},
  breaklines=true,
  breakatwhitespace=true,
  tabsize=3,
  otherkeywords={0,1,2,3,4,5,6,7,8,9},
  deletekeywords={data,frame,length,as,character,dunif,ps},
}

\begin{document} 

\section*{Hypotheses}

\noindent What features of the model do I consider when connecting it to biology?

\begin{itemize}

\item timing of the switch = onset of flowering

\item slope of the graded response = how quickly do you commit resources to flowering

\item response to different levels of interannual variation (i.e. how does the optimal control shift with increasing variance in season length) - has the flavor of comparing bifurcation diagrams - at what points do you get changes in qualitative behavior - not testing any particular number or trajectory but the qualitative changes tha occur as you tune parameters - when you shift from one type of behavior to another is a good way of comparing models

\item also think about when the constraints become active - what does it take to get yourself in a situation where those constraints are active - should they be doing that all the time vs. not

\item could compare constraint on rates vs. constraint on allocation (should be always active) vs. states

\end{itemize}

\section*{Development in the model}

\noindent How do I best represent the developmental decisions in the model? Is it possible that there are optimal control models for repressilators or oscillations that would be appropriate?

\begin{itemize}

\item timing of the switch = onset of flowering

\item slope of the graded response = how quickly do you commit resources to flowering

\item relative fitness of different strategies

\end{itemize}




\end{document}