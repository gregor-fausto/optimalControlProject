% !TEX TS-program = pdflatex
% !TEX encoding = UTF-8 Unicode

% This is a simple template for a LaTeX document using the "article" class.
% See "book", "report", "letter" for other types of document.

\documentclass[12pt]{article} % use larger type; default would be 10pt
\usepackage[utf8]{inputenc} % set input encoding (not needed with XeLaTeX)

%%% PAGE DIMENSIONS
\usepackage{geometry,amsmath,amssymb} % to change the page dimensions
\geometry{letterpaper} % or letterpaper (US) or a5paper or....
\usepackage{graphicx} % support the \includegraphics command and options

%%% END Article customizations

%%% The "real" document content comes below...

\title{Brief Article}
\author{The Author}
%\date{} % Activate to display a given date or no date (if empty),
         % otherwise the current date is printed 

\begin{document}

Let $\beta_1(t)$ be the per-capita rate of division by primary meristems, and $\beta_2(t)$ the per-capita
rate of division by inflorescence meristems. Let $u(t)$ be the fraction of primary meristem divisions 
producing one primary meristem and one vegetative meristem (type (a) in Figure 1); $1-u(t)$ is the fraction
of primary meristem divisions that are type (b) in Figure 1. The model is 

\begin{equation}
\begin{aligned}
\dot{P} & = -\beta_1(t)(1-u(t))P(t) \\
\dot{V} & = \beta_1(t)P(t) \\
\dot{I} & = \beta_1(t)(1-u(t))P(t) - \beta_2(t)I(t)\\
\dot{F} & = \beta_2(t)I(t) 
\end{aligned}
\end{equation}

subject to the constraints 
\begin{equation}
\begin{aligned}
& 0 \le \beta_1(t), \beta_2(t) \le M & \mbox{ Meristem constraint} \\
& 0 \le \beta_1(t)P(t) + \beta_2(t)I(t) \le V & \mbox{ Resource constraint}\\
& 0 \le u(t) \le 1
\end{aligned}
\end{equation}

The constraints on $\beta_1, \beta_2$ are as shown below. The square is the meristem constraint, the diagonal line is the 
resource constraint: $(\beta_1, \beta_2)$ needs to be below the line (which has infinite length but I've only drawn 
part of it). The meristem constraint is constant over time. The resource constraint line moves because it depends on the
values of $P(t),I(t)$ and $V(t)$. 

\smallskip 

\centerline{\includegraphics[width=5in]{ConstraintDiagram.jpg}}

I predict that when you work through the Hamiltonian etc., you will discover that increasing $\beta_1$ and/or $\beta_2$ is
always beneficial (i.e., this increases the Hamiltonian). In checking to see if this is true, you can assume that the 
$\lambda$'s are always positive, because that is implied in this model by their interpretation as shadow prices. 

If my prediction is right, then in Case 1 the optimal $\beta$s are always
on the portion of the resource constraint line inside the meristem constraint box; and 
in Case 2 the optimal $\beta$s are $(M,M)$, the top right corner of the meristem constraint. 






\end{document}
